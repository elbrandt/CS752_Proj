\documentclass[conference]{IEEEtran}

% Some very useful LaTeX packages include:
% (uncomment the ones you want to load)


\usepackage{cite}

% *** GRAPHICS RELATED PACKAGES ***
%
\ifCLASSINFOpdf
  % \usepackage[pdftex]{graphicx}
  % declare the path(s) where your graphic files are
  % \graphicspath{{../pdf/}{../jpeg/}}
  % and their extensions so you won't have to specify these with
  % every instance of \includegraphics
  % \DeclareGraphicsExtensions{.pdf,.jpeg,.png}
\else
  % or other class option (dvipsone, dvipdf, if not using dvips). graphicx
  % will default to the driver specified in the system graphics.cfg if no
  % driver is specified.
  % \usepackage[dvips]{graphicx}
  % declare the path(s) where your graphic files are
  % \graphicspath{{../eps/}}
  % and their extensions so you won't have to specify these with
  % every instance of \includegraphics
  % \DeclareGraphicsExtensions{.eps}
\fi
% graphicx was written by David Carlisle and Sebastian Rahtz. It is
% required if you want graphics, photos, etc. graphicx.sty is already
% installed on most LaTeX systems. The latest version and documentation
% can be obtained at: 
% http://www.ctan.org/pkg/graphicx
% Another good source of documentation is "Using Imported Graphics in
% LaTeX2e" by Keith Reckdahl which can be found at:
% http://www.ctan.org/pkg/epslatex
%
% latex, and pdflatex in dvi mode, support graphics in encapsulated
% postscript (.eps) format. pdflatex in pdf mode supports graphics
% in .pdf, .jpeg, .png and .mps (metapost) formats. Users should ensure
% that all non-photo figures use a vector format (.eps, .pdf, .mps) and
% not a bitmapped formats (.jpeg, .png). The IEEE frowns on bitmapped formats
% which can result in "jaggedy"/blurry rendering of lines and letters as
% well as large increases in file sizes.
%
% You can find documentation about the pdfTeX application at:
% http://www.tug.org/applications/pdftex





\usepackage{amsmath}

% *** SPECIALIZED LIST PACKAGES ***
%
%\usepackage{algorithmic}
% algorithmic.sty was written by Peter Williams and Rogerio Brito.
% This package provides an algorithmic environment fo describing algorithms.
% You can use the algorithmic environment in-text or within a figure
% environment to provide for a floating algorithm. Do NOT use the algorithm
% floating environment provided by algorithm.sty (by the same authors) or
% algorithm2e.sty (by Christophe Fiorio) as the IEEE does not use dedicated
% algorithm float types and packages that provide these will not provide
% correct IEEE style captions. The latest version and documentation of
% algorithmic.sty can be obtained at:
% http://www.ctan.org/pkg/algorithms
% Also of interest may be the (relatively newer and more customizable)
% algorithmicx.sty package by Szasz Janos:
% http://www.ctan.org/pkg/algorithmicx




% *** ALIGNMENT PACKAGES ***
%
%\usepackage{array}
% Frank Mittelbach's and David Carlisle's array.sty patches and improves
% the standard LaTeX2e array and tabular environments to provide better
% appearance and additional user controls. As the default LaTeX2e table
% generation code is lacking to the point of almost being broken with
% respect to the quality of the end results, all users are strongly
% advised to use an enhanced (at the very least that provided by array.sty)
% set of table tools. array.sty is already installed on most systems. The
% latest version and documentation can be obtained at:
% http://www.ctan.org/pkg/array


% IEEEtran contains the IEEEeqnarray family of commands that can be used to
% generate multiline equations as well as matrices, tables, etc., of high
% quality.




% *** SUBFIGURE PACKAGES ***
%\ifCLASSOPTIONcompsoc
%  \usepackage[caption=false,font=normalsize,labelfont=sf,textfont=sf]{subfig}
%\else
%  \usepackage[caption=false,font=footnotesize]{subfig}
%\fi
% subfig.sty, written by Steven Douglas Cochran, is the modern replacement
% for subfigure.sty, the latter of which is no longer maintained and is
% incompatible with some LaTeX packages including fixltx2e. However,
% subfig.sty requires and automatically loads Axel Sommerfeldt's caption.sty
% which will override IEEEtran.cls' handling of captions and this will result
% in non-IEEE style figure/table captions. To prevent this problem, be sure
% and invoke subfig.sty's "caption=false" package option (available since
% subfig.sty version 1.3, 2005/06/28) as this is will preserve IEEEtran.cls
% handling of captions.
% Note that the Computer Society format requires a larger sans serif font
% than the serif footnote size font used in traditional IEEE formatting
% and thus the need to invoke different subfig.sty package options depending
% on whether compsoc mode has been enabled.
%
% The latest version and documentation of subfig.sty can be obtained at:
% http://www.ctan.org/pkg/subfig




% *** FLOAT PACKAGES ***
%
%\usepackage{fixltx2e}
% fixltx2e, the successor to the earlier fix2col.sty, was written by
% Frank Mittelbach and David Carlisle. This package corrects a few problems
% in the LaTeX2e kernel, the most notable of which is that in current
% LaTeX2e releases, the ordering of single and double column floats is not
% guaranteed to be preserved. Thus, an unpatched LaTeX2e can allow a
% single column figure to be placed prior to an earlier double column
% figure.
% Be aware that LaTeX2e kernels dated 2015 and later have fixltx2e.sty's
% corrections already built into the system in which case a warning will
% be issued if an attempt is made to load fixltx2e.sty as it is no longer
% needed.
% The latest version and documentation can be found at:
% http://www.ctan.org/pkg/fixltx2e


%\usepackage{stfloats}
% stfloats.sty was written by Sigitas Tolusis. This package gives LaTeX2e
% the ability to do double column floats at the bottom of the page as well
% as the top. (e.g., "\begin{figure*}[!b]" is not normally possible in
% LaTeX2e). It also provides a command:
%\fnbelowfloat
% to enable the placement of footnotes below bottom floats (the standard
% LaTeX2e kernel puts them above bottom floats). This is an invasive package
% which rewrites many portions of the LaTeX2e float routines. It may not work
% with other packages that modify the LaTeX2e float routines. The latest
% version and documentation can be obtained at:
% http://www.ctan.org/pkg/stfloats
% Do not use the stfloats baselinefloat ability as the IEEE does not allow
% \baselineskip to stretch. Authors submitting work to the IEEE should note
% that the IEEE rarely uses double column equations and that authors should try
% to avoid such use. Do not be tempted to use the cuted.sty or midfloat.sty
% packages (also by Sigitas Tolusis) as the IEEE does not format its papers in
% such ways.
% Do not attempt to use stfloats with fixltx2e as they are incompatible.
% Instead, use Morten Hogholm'a dblfloatfix which combines the features
% of both fixltx2e and stfloats:
%
% \usepackage{dblfloatfix}
% The latest version can be found at:
% http://www.ctan.org/pkg/dblfloatfix

% *** PDF, URL AND HYPERLINK PACKAGES ***
%
\usepackage{url}
% Basically, \url{my_url_here}.


% for comments
\usepackage{color}
\newcommand{\mycomment}[3]{\emph{\textcolor{#2}{#1}: }\textcolor{#2}{#3}}
\newcommand{\todo}[1]{\mycomment{Todo}{red}{#1}}
\newcommand{\sangeetha}[1]{\mycomment{Sangeetha}{magenta}{#1}}
\newcommand{\eric}[1]{\mycomment{Eric}{cyan}{#1}}

% correct bad hyphenation here
\hyphenation{op-tical net-works semi-conduc-tor}


\begin{document}
% paper title
\title{Perceptron-Based Prefetch Filtering in Gem5\\ \Large{CS752 Project}}


% author names and affiliations
% use a multiple column layout for up to three different
% affiliations
\author{
\IEEEauthorblockN{Eric Brandt}
\IEEEauthorblockA{University of Wisconsin, Madison\\
Email: \texttt{elbrandt@wisc.edu}}
\and
\IEEEauthorblockN{Sangeetha Grama Srinivisan}
\IEEEauthorblockA{University of Wisconsin, Madison\\
Email: \texttt{sgsrinivasa2@wisc.edu}}}


% make the title area
\maketitle

\section{Introduction}
Prefetching has been used as an effective strategy to improve processor performance. Many prefetching strategies are proposed based on spatial or temporal access patterns observed in the program. Machine learning algorithms ranging from simple perceptrons to complex LSTM (Long-Short-Term-Memory) models are leveraged to design and optimise prefetching strategies. One such design uses a perceptron-based prefetch filter\cite{ppf} to enhance Signature Path Prefetching \cite{SPP}, a lookahead prefetcher. In this project we aim to implement this perceptron-based prefetch filter\cite{ppf} in GEM5 \cite{lowepower2020gem5} simulator and evaluate the performance implications of the filtered prefetches on a wide range of workloads. 

\section{Relevance}
Memory accesses span over multiple cycles. The rate of improvement in memory access speeds compared to the rate of improvement in processor speeds indicates that there is a wide gap known as the Memory Wall \cite{mem-wall}. This significant difference in speeds can be mitigated by prefetching memory into the cache. The prefetching strategies are continuously optimised to improve accuracy of prefetching while also maintaining a good coverage, i.e, the prefetcher should be able to eliminate cache misses and all the prefetched memory should be used by the processor. While the former requirement focuses on improving processor performance, the latter requirement helps conserve memory bandwidth and the space availability in the cache. Machine learning algorithms have been used to improve or enhance prefetching, like, offline parameter optimisation for baseline prefetchers or on-line parameter tuning to provide feedback to the prefetcher based on the result of the prefetch, such as the perceptron-based prefetch filter\cite{ppf}. Implementing this filter in GEM5 will help in getting a detailed analysis of performance implications on a wide range of program workloads and architectures.  Further, the prefetch filter can be used as a standalone unit with any baseline prefetcher, giving researchers the opportunity to explore the use of this filter with their designs.

\section{Evaluation Methods}

Implementation success will be measured both qualitatively and quantitatively. The \textit{qualitative} evaluation will verify that the Perceptron Prefetcher algorithm is implemented \textit{correctly} (e.g, behaves in a manner consistent with the proposed design, and does not contain bugs). This will be achieved through a combination of custom log files generated during runtime that are inspected post-mortem, real-time inspection within the debugger itself, and carefully crafted sample workloads designed to elicit certain behaviors of the prefetcher to exercise all possible code paths. The \textit{quantitative} evaluation will involve benchmarks run on our prefetcher implementation as well as existing prefetcher strategies for comaprison purposes. For these benchmarks, we hope to use industry standard benchmarks such as SPEC CPU 2017 used by \cite{ppf} and SPEC2006 suite as used by \cite{sandbox}. Assuming we are able to run these benchmarks in Gem5 and use the same warmup and measured code sections as those authors did, we may be able to very roughly compare performance numbers with their stated results. 

\section{References}

Pugsley et.al present a prefetching technique known as Sandbox Prefetching\cite{sandbox}. The design first uses global pattern confirmation of the cache accesses using a bloom filter and once the pattern is confirmed, can immediately start prefetching addresses using the confirmed pattern. The sandbox is used to prevent unwanted memory accesses and conserve memory bandwidth and pages available in the cache. The sandbox is used to match the program’s cache access patterns with one of the candidate patterns (each with a different offset). The candidate pattern for which most of the memory accesses match is treated as the confirmed pattern and prefetches are issued using this pattern. While the design reduces the prefetch aggressiveness and uses spatial locality, it does not take into account the program order for prefetching memory.

Signature Path Prefetching \cite{SPP} (SPP) is a type of a lookahead prefetcher where program order is used. SPP is designed to not only predict the memory location accessed but also the order of the access within a given page. The method predicts the future memory access patterns without requiring inputs from the PC or the branch predictor. The design has 3 components - Signature table, Pattern table and the Prefetch engine. The signature table stores the physical page numbers, the previously accessed block in that page along with a hash of the previous access patterns to that page, compressed as a signature. The pattern table, indexed using this signature, is used to store future stride access patterns and the confidence for each of the prospective prefetch patterns. The pattern table is indexed by the signature, unlike the signature table which is indexed using the physical page number since multiple pages can have the same memory access pattern. The prefetch engine issues the prefetch pattern which has a confidence greater than a threshold (known as filtering) and also performs lookahead by using a lookahead signature. This lookahead signature is generated from the old signature and used to access the pattern table again to look further ahead and find more prefetch candidates, repeating the prefetching process. Though the design is simple and performs better than spatial prefetchers like AMPM (Access Map Pattern Matching prefetcher), the effectiveness of the design is dependent on the threshold used to determine if a prefetch pattern with a certain confidence value can be issued.

Eshan et. al propose a method of using a perceptron-based prefetching filter \cite{ppf} to control the aggressiveness of prefetchers. The method is implemented using SPP\cite{SPP} as the baseline prefetcher. Since in SPP, the threshold for confidence is used to determine whether a prefetch needs to be issued, using a perceptron to make this decision helps in 2 aspects. First, it provides a generalisation of the process used in SPP to decide to issue a prefetch and secondly, it helps to decide which level of memory to prefetch into based on the output of the perceptron. The SPP method in \cite{SPP} reserved a part of the L2 cache for prefetching while the perceptron-based prefetching filter allows prefetching to be done either to L2 or the next cache level. The filter acts as a check to control the aggressiveness of the prefetcher by maintaining 3 tables - weight table, prefetch table and a reject table. The weight table maintains the weight for each feature (input to the perceptron) and is used to compute the weighted sum of the features - the confidence level of the prefetch, which is then compared against a threshold. The prefetches that have a confidence level higher than the threshold are stored in the prefetch table, while those that do not are stored in the reject table, both of which are used to tune the weights in the perceptron. Training is done when an address from the memory request is found in either of the tables, indicating a correct or a misprediction following which the weights are updated accordingly. This design uses the perceptron-based prefetch filter as a stand-alone module that can be used with any type of baseline prefetcher, and thus, can enhance any existing prefetcher. 

\section{Project Plan}

Our plan, broadly, is to implement and evaluate a Perceptron-based Cache Prefetch strategy based on the design of \cite{ppf} within the Gem5 simulator. More specifically, we can break this plan into a series of tasks that will lead us to our goal. Because we cannot predict all of the obstacles we may encounter in the implementation, we will consider each successfully completed task as a worthwhile effort towards the full plan. To this end, we have developed the following list of tasks, and planned their completion around the deadlines under which we are working. We divide our tasks into milestones: \begin{itemize}
\item\textbf{Milestone 1}: Tasks that we plan to complete before the progress report deadline:
  \begin{enumerate}
    \item Survey the prefetch mechanisms that exist within Gem5 today. We will gain understanding of the implementation of cache prefetch mechanisms by studying of the files in \texttt{src/mem/cache/prefetch}, with careful attention paid to \texttt{signature\_path\_v2.cc}, and the inheritance hierarchy of classes within that folder.
    \item Continue a more comprehensive literature review of current cache prefetch strategies to better understand the lineage of the Preceptron Prefetch design. Particular attention will be given to studying the design of Signature Path Prefetching \cite{SPP}, as this is the groundwork on which Perceptron-Based Prefetch Filtering is built.
    \item Develop/curate a set of benchmarks and statistics on which to evaluate various prefetch strategies. This includes development of scripts to run the benchmarks in Gem5 as well as analyze the generated statistics.
    \item Create a new trivial-capability C++ prefetch object within Gem5, likely derived from \texttt{Prefetcher::Base} or \texttt{Prefetcher::Queued}, with associated Python wrappers, exposing settings that allow the prefetcher to be tuned via Gem5 Python system definition scripts.
    \item Test the framework of the new trivial prefetcher by ensuring our benchmarks can be executed on simulation systems defined to use this prefetcher within the memory heirarchy.
    \item Prepare and submit a mid-project progress report by the November 13 deadline.
  \end{enumerate}
\item\textbf{Milestone 2}: Tasks that we plan to complete before the Lightning Talk deadline
  \begin{enumerate}
      \item Implement the Perceptron Prefetcher described in \cite{ppf} by modifying the trivial prefetcher created in Milestone 1.
      \item Develop test programs for execution in Gem5 that are both simple, and specifically engineered to test the correctness of the implementation of the Perceptron Prefetcher.
      \item Benchmark our implementation to test both for correctness of implementation, and for performance using the scripts developed in Milestone 1.
      \item Compare performance of key statistics with other prefetch designs that already exist in Gem5.
      \item Prepare a `Lightning-talk' to evangelize the work and results involved in this project by December 2.
      \item Write a `research paper'-quality report of our work and the results we measure by December 15.
  \end{enumerate} 
\item\textbf{Stretch Milestones}: Tasks that we hope to complete, but may be too large to fully complete within the time constraints of the project.
  \begin{enumerate}
      \item Increase the number of `tunable' settings exposed by our Perceptron Prefetcher, to allow for more experimentation. 
      \item Perform a more exhaustive evaluation of the effect of the available settings of our prefetcher with various different cache configurations in simulated systems.
      \item Test different workloads to try to discern the particular conditions and code characteristics that favor the Perceptron Prefetcher as compared to other strategies. It may be interesting to run different prefetch strategies on physics-based-animation/simulation workloads, to compare prefetch performance in generalized benchmarks (e.g. SPEC CPU) with the numerical, highly-regular access patterns of this type of domain-specific code.
      \item Evaluate our implemented prefetch filter on prefetch strategies other than Signature Prefetch, to ascertain the transferability of a prefetch filter to other prefetch strategies.
      \item Propose ways to further improve the design of the Perceptron Prefetcher, or at least define roadmaps for future research.
  \end{enumerate}
\end{itemize}

%A plan to address other related work.

% An example of a floating figure using the graphicx package.
% Note that \label must occur AFTER (or within) \caption.
% For figures, \caption should occur after the \includegraphics.
% Note that IEEEtran v1.7 and later has special internal code that
% is designed to preserve the operation of \label within \caption
% even when the captionsoff option is in effect. However, because
% of issues like this, it may be the safest practice to put all your
% \label just after \caption rather than within \caption{}.
%
% Reminder: the "draftcls" or "draftclsnofoot", not "draft", class
% option should be used if it is desired that the figures are to be
% displayed while in draft mode.
%
%\begin{figure}[!t]
%\centering
%\includegraphics[width=2.5in]{myfigure}
% where an .eps filename suffix will be assumed under latex, 
% and a .pdf suffix will be assumed for pdflatex; or what has been declared
% via \DeclareGraphicsExtensions.
%\caption{Simulation results for the network.}
%\label{fig_sim}
%\end{figure}

% Note that the IEEE typically puts floats only at the top, even when this
% results in a large percentage of a column being occupied by floats.


% An example of a double column floating figure using two subfigures.
% (The subfig.sty package must be loaded for this to work.)
% The subfigure \label commands are set within each subfloat command,
% and the \label for the overall figure must come after \caption.
% \hfil is used as a separator to get equal spacing.
% Watch out that the combined width of all the subfigures on a 
% line do not exceed the text width or a line break will occur.
%
%\begin{figure*}[!t]
%\centering
%\subfloat[Case I]{\includegraphics[width=2.5in]{box}%
%\label{fig_first_case}}
%\hfil
%\subfloat[Case II]{\includegraphics[width=2.5in]{box}%
%\label{fig_second_case}}
%\caption{Simulation results for the network.}
%\label{fig_sim}
%\end{figure*}
%
% Note that often IEEE papers with subfigures do not employ subfigure
% captions (using the optional argument to \subfloat[]), but instead will
% reference/describe all of them (a), (b), etc., within the main caption.
% Be aware that for subfig.sty to generate the (a), (b), etc., subfigure
% labels, the optional argument to \subfloat must be present. If a
% subcaption is not desired, just leave its contents blank,
% e.g., \subfloat[].


% An example of a floating table. Note that, for IEEE style tables, the
% \caption command should come BEFORE the table and, given that table
% captions serve much like titles, are usually capitalized except for words
% such as a, an, and, as, at, but, by, for, in, nor, of, on, or, the, to
% and up, which are usually not capitalized unless they are the first or
% last word of the caption. Table text will default to \footnotesize as
% the IEEE normally uses this smaller font for tables.
% The \label must come after \caption as always.
%
%\begin{table}[!t]
%% increase table row spacing, adjust to taste
%\renewcommand{\arraystretch}{1.3}
% if using array.sty, it might be a good idea to tweak the value of
% \extrarowheight as needed to properly center the text within the cells
%\caption{An Example of a Table}
%\label{table_example}
%\centering
%% Some packages, such as MDW tools, offer better commands for making tables
%% than the plain LaTeX2e tabular which is used here.
%\begin{tabular}{|c||c|}
%\hline
%One & Two\\
%\hline
%Three & Four\\
%\hline
%\end{tabular}
%\end{table}


% Note that the IEEE does not put floats in the very first column
% - or typically anywhere on the first page for that matter. Also,
% in-text middle ("here") positioning is typically not used, but it
% is allowed and encouraged for Computer Society conferences (but
% not Computer Society journals). Most IEEE journals/conferences use
% top floats exclusively. 
% Note that, LaTeX2e, unlike IEEE journals/conferences, places
% footnotes above bottom floats. This can be corrected via the
% \fnbelowfloat command of the stfloats package.

% conference papers do not normally have an appendix


% use section* for acknowledgment
%\section*{Acknowledgment}
%The authors would like to thank...

% trigger a \newpage just before the given reference
% number - used to balance the columns on the last page
% adjust value as needed - may need to be readjusted if
% the document is modified later
%\IEEEtriggeratref{8}
% The "triggered" command can be changed if desired:
%\IEEEtriggercmd{\enlargethispage{-5in}}

% references section
\bibliographystyle{IEEEtran}
\bibliography{IEEEabrv,ref_proposal}

% that's all folks
\end{document}

